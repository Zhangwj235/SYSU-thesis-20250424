% !TEX encoding = UTF-8 Unicode
\documentclass[doctype=proposal,printmode=final]{sysuthesis}

% 设置论文的基本信息,包括题目、作者、专业、导师、学院、摘要和关键词等必要信息
% !TEX root = ./thesis.tex

% 设置基本信息
\sysuset{
    %******************************
    % 注意:配置里面不要出现空行,请删掉空白行或者添加注释符号
    %******************************
    % 中文姓名
    author = {张三},
    % 英文姓名
    author-en = {Zhang San},
    % 中文标题
    title = {中山大学研究生学位论文\LaTeX{}非官方模版},
    % 英文标题
    title-en = {English Title for Sun Yat-sen University Thesis \LaTeX{} Unofficial Template},
    % 中文关键词
    keywords = {天体物理;引力波;黑洞},
    % 英文关键词
    keywords-en = {Astrophysics; Gravitational Wave; Black Holes},
    % 学号
    student-id = {22110000},
    % 专业
    major = {天体物理},
    % 专业英文
    major-en = {Astrophysics},
    % 指导教师姓名
    supervisor = {李四 \quad 副教授},
    % 指导教师英文姓名
    supervisor-en = {Associate Professor \quad Li Si},
    % 学科门类
    discipline = {理学},
    % 学位
    degree = {博士},
    % 校区
    campus = {珠海},
    % 学院
    school = {物理与天文学院},
    % 研究团队
    research-team = {天文团队},
    % 日期
    date = {2022年6月},
}

% 设置模板的一些参数
\sysuset{
    %******************************
    % 注意:配置里面不要出现空行,请删掉空白行或者添加注释符号
    %******************************
    % % 致谢环境名
    % ack-name = {后记},
    % % 附录环境名
    % appendix-name = {附录},
    % % 参考文献名
    % bib-name = {参考文献},
    % % 参考文献表样式,可选项为 numerical(顺序编码制)、author-year(著者-出版年制)
    % bib-style = author-year,
    % 图表等标签和标题的分隔符号,可选项为 none(取消符号)、colon (冒号)、period(句点)、space(空格)、quad(空白)、newline(换行)
    caption-labelsep = space,
    % 算法环境标签和标题的分隔符号
    algorithm-labelsep = {:},
    % % 结论环境名
    % conclusion-name = {结论},
    % % 目录名
    % contents-name = {目\qquad 录},
    % 本模版的主要颜色
    main-color = sysugreen,
    % % 索引名
    % index-name = {索引},
    % % 插图索引名
    % listfigure-name = {插图索引},
    % % 表格索引名
    % listtable-name = {表格索引},
    % % 公式索引名
    % list-equation-name = {公式索引},
    % % 插图标签名
    % figure-name = {图},
    % % 表格标签名
    % table-name = {表},
    % % 插图索引中插图列表的标签名
    % figure-prename-in-lof = {图},
    % % 表格索引中表格列表的标签名
    % table-prename-in-lot = {表},
    % % 插图索引中标签名与标题的距离
    % indent-in-lof = {3.25em},
    % % 表格索引中标签名与标题的距离
    % indent-in-lot = {3.25em},
    % % lstlisting 环境索引名
    % listoflistings-name = {代码索引},
    % % lstlisting 环境标签名
    % lstlisting-name = {代码},
    % % lstlisting 环境的风格,可以查看 listings 宏包的文档自行设置
    % lst-style = {codestyle},
    % % 长表格续表的表头
    % longtable-continued-head = {(续)},
    % % 长表格的表尾
    % longtable-continued-foot = {下页续},
}

% % 图表等标题的对齐方式,可选项为 justified(两端对齐)、centering(居中对齐)、raggedright(左对齐标题)、raggedleft(右对齐标题),详情请看 caption 宏包
% \captionsetup{justification=justified}

% 定义所有的图片文件在 figures 子目录下
\graphicspath{{./figures/}}

% 设置新的latex命令
\newcommand{\dd}{~\mathrm{d}}

% 调用的宏包
\usepackage{longtable}
\usepackage{tablefootnote}
\usepackage{hologo}
\usepackage[edges]{forest}
\forestset{%
    directory/.style={%
        for tree={%
            edge+=thick, 
            folder, 
            font=\ttfamily\zihao{-5},
            grow'=0,
            draw,
        },
    },
}

\usepackage{multirow}

\begin{document}

% 前置部分
\frontmatter

% 扉页
\maketitle

% 目录
\tableofcontents

% 主体部分
\mainmatter

% 正文

\section*{说\hspace*{2\ccwd}明}

\begin{itemize}
    \item[一、]开题报告应包括下列主要内容,见目录。
    \item[二、]开题报告字数应不少于1.5万字。
    \item[三、]开题报告最迟应于第四学期结束前完成。
    \item[四、]若本次开题报告未通过,需在6-12个月内再次进行开题报告。第二次学位论文开题报告仍未通过者,按评议小组的建议进行后续分流工作。
    \item[五、]开题报告结束后,评议小组要填写《博士学位论文开题报告评议结果》,与主席签字的“博士学位论文开题答辩原始记录”一同上报学院研究生教学秘书备案。学生需将修改过的《开题报告》和《博士学位论文开题报告修改情况确认表》于开题答辩后一周内上交学院研究生教学秘书备案。
    \item[六、]字体、字号及其他规定
    
    论文中所用中文字体(除各级标题外)为宋体,各级标题用黑体;论文中所用数字、英文为新罗马字体。

    \begin{tabular}{ll}
        节标题         &小3号字,建议段前0.5行,段后0.5行;\\
        条标题         &4号字,建议段前0.5行,段后0.5行;\\
        款、项标题      &小4号字, 建议段前0行,段后0行;\\
        正文           &小4号字,建议段前0行,段后0行,每页约33行。
    \end{tabular}
    \item[七、]层次代号及说明
    \par\ 
    \begin{table}[h]
        \zihao{5}
        \centering
        \begin{tabular}{p{0.1\linewidth}|p{0.45\linewidth}|p{0.35\linewidth}}
            \hline 
            层次名称 & 示例 & 说明 \\
            \hline 
            节 &1~$\Box\Box\cdots\cdots\Box$ & \multirow{3}*{题序顶格书写,阐述内容另起一段} \\
            \cline{1-2}
            条 &1.1~$\Box\Box\cdots\cdots\Box$ & \\
            \cline{1-2}
            款 &1.1.1~$\Box\Box\cdots\cdots\Box$  & \\
            \hline
            项 &\hspace*{2\ccwd}(1)~$\Box\Box\cdots\cdots\Box$ \hspace*{2\ccwd}$\Box\Box\cdots\cdots\Box\Box$  & 题序空4个半角字符书写,内容空4个半角字符接排 \\
            \hline
        \end{tabular}
    \end{table}
    \item[八、]常用的四种参考文献类型标注形式。例如:
    \begin{enumerate}
        \item 这是一个期刊的引用\cite{LIGOScientific:2017zic};
        \item 这是一个图书的引用\cite{Rubakov:2017xzr,Zhang:2021};
        \item 这是一个研讨会论文的引用\cite{Tanikawa:2021+x};
        \item 这是博士论文的引用\cite{Migenda:2019xbm,HuangGuoYuan:2020},这是硕论文的引用\cite{Shojaeifar:2015csv,SongRen:2020};
        \item 这是电子文献的引用\cite{Piro:2021oaa,bilibili:read}。
        \item 这是报纸的引用\cite{Li:2005}。
    \end{enumerate}
\end{itemize}

    
\section{课题来源及研究的目的和意义}
\subsection{课题来源或研究背景}
\subsection{研究的目的及意义}
(不少于1000字)

\section{国内外在该方向的研究现状及分析}
(文献综述)
\subsection{国外研究现状}
\subsection{国内研究现状}
(注意对所引用国内外文献的准确标注)
\subsection{国内外文献综述的简析}
(不少于1000字)
(综合评述:国内外研究取得的成果,存在的不足或有待深入研究的问题)

\section{学位论文的主要研究内容、实施方案及其可行性论证前期研究与论证工作的结果}
\subsection{主要研究内容}
(不少于2000字)
(撰写宜使用将来时态,不能只列出论文目录来代替对研究内容的分析论述)
\subsection{实施方案及其可行性论证}
(不少于3000字)

\section{已完成的研究工作}
(详细撰写目前已进行的研究工作内容和完成情况)

\section{论文进度安排,预期达到的目标}
\subsection{进度安排}
(从确定博士选题收集文献写起)
\subsection{预期达到的目标}

\section{学位论文预期创新点}
(要根据研究内容和国内外研究现状准确提炼,充分体现创新性)

\section{为完成课题已具备和所需的条件、外协计划及经费}

\section{预计研究过程中可能遇到的困难、问题,以及解决的途径}

\section{主要参考文献}
(应在50篇以上,其中外文资料不少于二分之一,参考文献中近五年(从开题当年算起)内发表的文献一般不少于三分之一,且必须有近二年内发表的文献资料)。



\makebib{refs}

\cleardoublepage
\end{document}