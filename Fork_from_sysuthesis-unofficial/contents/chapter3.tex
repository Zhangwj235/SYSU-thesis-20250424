\chapter{更新描述}
\chapteren{Update Description}

\section*{\texttt{v1.1.6 2025/04/10}}
\begin{itemize}
    \item 增加摘要环境的 \texttt{noinfo} 选项,允许用户在摘要中不显示作者信息。
    \item 修复 $\hbar$ 输出错误问题,修复参考文献的引用标注排序问题。
    \item 更新查重和盲审模式下的输出,查重模式保留附录,盲审模式保留学校信息。
    \item 更新 VSCode 的 \texttt{latex-workshop} 插件的推荐配置,使用王然老师的配置\footnote{见 \url{https://github.com/OsbertWang/install-latex-guide-zh-cn}}。
\end{itemize}

\section*{\texttt{v1.1.5 2024/04/12}}
\begin{itemize}
    \item 增加了 \texttt{\char92 blindthis[<replace>]\{<content>\}} 命令,用于在盲审模式下(即 \texttt{printmode=blindmode})替换内容。在非盲审模式下,该命令不会有任何效果。在盲审模式下,该命令会将 \texttt{<content>} 替换为 \texttt{<replace>};如果直接使用 \texttt{\char92 blindthis\{<content>\}} 命令,则内容不会打印出来。
    \item 微调 \texttt{publications} 和 \texttt{achievements} 环境中标签的左边距。
\end{itemize}

\section*{\texttt{v1.1.4 2024/04/10}}
\begin{itemize}
    \item 修复了附录章在英文模式下的标签名,从 Chapter 改为 Appendix。并且新增了两个无标签章的命令 \texttt{\char92 notagchapter} 和 \texttt{\char92 notagchapteren},用于不带章节标签的章节。
    \item 对图表标题的对齐方式进行了调整,使之居中,并将之前第二标题的额外空格去除。
    \item 完善 \texttt{algorithm2e} 标题的格式,使之与图表标题一致。
\end{itemize}

\section*{\texttt{v1.1.3 2024/03/30}}
\begin{itemize}
    \item 在模板参数中对 \texttt{gbt7714} 宏包进行参数传递,可以使用著者-出版年制的参考文献格式。
    \item 使用 \texttt{algorithm2e} 宏包定制算法环境。
    \item 增加 \texttt{language} 选项,可选 \texttt{chinese} | \texttt{english} ( \texttt{<default> = chinese}),或 \texttt{zh} | \texttt{en}。如果选项为 \texttt{language=english}(或 \texttt{language=en}), 这将会将章节图表等的标题语言设置为英文。
    \item 将 \texttt{\char92 info} 命令改为 \texttt{\char92 sysuset},对模版的一些参数(如图表标签名和 \\ \texttt{acknowledgements} 环境名称等)任由用户自定义,详情见 \texttt{setup.tex} 文件。
    \item \texttt{openany}、\texttt{openright} 和 \texttt{fontset} 为 \texttt{ctexbook} 文档类的选项,不应作为模板的选项,现已移除。
    \item 撤销将 \texttt{\char92 ref} 命令的引用格式重设为 \texttt{(\char92 autoref\{key\})} 的更改。
    \item 解决了一些与 \texttt{hyperref} 宏包的冲突问题。
\end{itemize}

\section*{\texttt{v1.1.2 2024/03/14}}
\begin{itemize}
    \item 放弃自制的 \texttt{sysuthesis.bst},改用 \texttt{gbt7714} 宏包。
    \item 增加  \texttt{count\_chinese.py} Python 脚本,用于统计中文字数。
    \item 重新设置论文信息的设置方式,即键值对(key-value)的格式,更加友好。
    \item 修改了 \texttt{checkmode}的版面,去除无效的空白页。
    \item 添加了中山大学的颜色 \texttt{sysugreen}、 \texttt{sysured} 和 \texttt{spablue}。
    \item 给出了长表格的示例,并配置了 \texttt{tabularray} 的风格。
\end{itemize}

\section*{\texttt{v1.1.1 2023/03/30}}
\begin{itemize}
    \item 使用 \texttt{\char92 raggedbottom} 调整页面的垂直对齐方式, 当页面内容不足时, 这将减少页面顶部和底部之间的间距,使得页面看起来更加紧凑。
    \item 增加 \texttt{fontset} 选项 ( \texttt{<default> = fandol}),指定\CTeX{}宏集加载的字库,详情请查看\CTeX{}宏集的具体说明。例如,如果您的系统为Windows,则可以用以下选项:
\begin{lstlisting}[language=TeX]
\documentclass[doctype=thesis,printmode=final,openright,blankleft,fontset=windows]{sysuthesis}
\end{lstlisting}
    如果您在 Overleaf 上编译,则可以设置为:
\begin{lstlisting}[language=TeX]
\documentclass[doctype=thesis,printmode=final,openright,blankleft,fontset=ubuntu]{sysuthesis}
\end{lstlisting}
    目前 Mac OS 可以暂时使用 \texttt{fontset=macnew},依然解决不了找不到对应字体的警告问题,但无伤大雅。
    \item 对一些笔误进行了修改。
\end{itemize}

\section*{\texttt{v1.1.0 2023/03/03}}
\begin{itemize}
    \item 增加以下模版选项:
    \begin{itemize}
        \item \texttt{doctype},可选 \texttt{thesis}| \texttt{proposal} ( \texttt{<default> = thesis}),分别为学位论文和开题报告的格式。
        \item \texttt{printmode},可选 \texttt{final}| \texttt{checkmode}| \texttt{blindmode} ( \texttt{<default> = final}),分别为终稿、查重和盲审的打印模式。
        \item \texttt{openright}| \texttt{openany},互为 \texttt{true}| \texttt{false} ( \texttt{<default> = openright})。\\ \texttt{openright} 选项为每一章在右页(奇数页)开始, \texttt{openright} 选项为在上一章结束的下一页开始。
        \item \texttt{blankleft} ( \texttt{<default> = false}),当 \texttt{blankleft = true} 时,章节结束的偶数页如果没有内容,使之空白,但页码计数器仍然有效。
    \end{itemize}
    \item 增加了 \texttt{appendixenv}、 \texttt{publications}和 \texttt{achievements} 环境,分别为附录、学术论文发表列表和学术成果列表的环境。
    \item 对论文扉页进行了微调。
    \item 修改 \texttt{lstlisting} 双语标题格式,微调相关颜色。
    \item 增加了 NASA/ADS Export Citation 的期刊名命令,不需要再手动修改以避免 \hologo{BibTeX} 编译出错。
\end{itemize}

\section*{\texttt{v1.0.1 2022/03/06}}
\begin{itemize}
    \item 最新适配物理与天文学院的格式要求,调整了参考文献的引用格式并添加文献类型标识,将中文与西文之间的一个半角字符的自动间距关闭。 \texttt{\char92 texttt}命令只在本文档用以展示命令,不建议大家使用。
\end{itemize}

\section*{\texttt{v1.0 2022/02/23}}
\begin{itemize}
    \item 最初版本。
\end{itemize}