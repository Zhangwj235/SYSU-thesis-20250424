% !TEX root = ./thesis.tex

% 设置基本信息
\sysuset{
    %******************************
    % 注意:配置里面不要出现空行,请删掉空白行或者添加注释符号
    %******************************
    % 中文姓名
    author = {张三},
    % 英文姓名
    author-en = {Zhang San},
    % 中文标题
    title = {中山大学研究生学位论文\LaTeX{}非官方模版},
    % 英文标题
    title-en = {English Title for Sun Yat-sen University Thesis \LaTeX{} Unofficial Template},
    % 中文关键词
    keywords = {天体物理;引力波;黑洞},
    % 英文关键词
    keywords-en = {Astrophysics; Gravitational Wave; Black Holes},
    % 学号
    student-id = {22110000},
    % 专业
    major = {天体物理},
    % 专业英文
    major-en = {Astrophysics},
    % 指导教师姓名
    supervisor = {李四 \quad 副教授},
    % 指导教师英文姓名
    supervisor-en = {Associate Professor \quad Li Si},
    % 学科门类
    discipline = {理学},
    % 学位
    degree = {博士},
    % 校区
    campus = {珠海},
    % 学院
    school = {物理与天文学院},
    % 研究团队
    research-team = {天文团队},
    % 日期
    date = {2022年6月},
}

% 设置模板的一些参数
\sysuset{
    %******************************
    % 注意:配置里面不要出现空行,请删掉空白行或者添加注释符号
    %******************************
    % % 致谢环境名
    % ack-name = {后记},
    % % 附录环境名
    % appendix-name = {附录},
    % % 参考文献名
    % bib-name = {参考文献},
    % % 参考文献表样式,可选项为 numerical(顺序编码制)、author-year(著者-出版年制)
    % bib-style = author-year,
    % 图表等标签和标题的分隔符号,可选项为 none(取消符号)、colon (冒号)、period(句点)、space(空格)、quad(空白)、newline(换行)
    caption-labelsep = space,
    % 算法环境标签和标题的分隔符号
    algorithm-labelsep = {:},
    % % 结论环境名
    % conclusion-name = {结论},
    % % 目录名
    % contents-name = {目\qquad 录},
    % 本模版的主要颜色
    main-color = sysugreen,
    % % 索引名
    % index-name = {索引},
    % % 插图索引名
    % listfigure-name = {插图索引},
    % % 表格索引名
    % listtable-name = {表格索引},
    % % 公式索引名
    % list-equation-name = {公式索引},
    % % 插图标签名
    % figure-name = {图},
    % % 表格标签名
    % table-name = {表},
    % % 插图索引中插图列表的标签名
    % figure-prename-in-lof = {图},
    % % 表格索引中表格列表的标签名
    % table-prename-in-lot = {表},
    % % 插图索引中标签名与标题的距离
    % indent-in-lof = {3.25em},
    % % 表格索引中标签名与标题的距离
    % indent-in-lot = {3.25em},
    % % lstlisting 环境索引名
    % listoflistings-name = {代码索引},
    % % lstlisting 环境标签名
    % lstlisting-name = {代码},
    % % lstlisting 环境的风格,可以查看 listings 宏包的文档自行设置
    % lst-style = {codestyle},
    % % 长表格续表的表头
    % longtable-continued-head = {(续)},
    % % 长表格的表尾
    % longtable-continued-foot = {下页续},
}

% % 图表等标题的对齐方式,可选项为 justified(两端对齐)、centering(居中对齐)、raggedright(左对齐标题)、raggedleft(右对齐标题),详情请看 caption 宏包
% \captionsetup{justification=justified}

% 定义所有的图片文件在 figures 子目录下
\graphicspath{{./figures/}}

% 设置新的latex命令
\newcommand{\dd}{~\mathrm{d}}

% 调用的宏包
\usepackage{longtable}
\usepackage{tablefootnote}
\usepackage{hologo}
\usepackage[edges]{forest}
\forestset{%
    directory/.style={%
        for tree={%
            edge+=thick, 
            folder, 
            font=\ttfamily\zihao{-5},
            grow'=0,
            draw,
        },
    },
}