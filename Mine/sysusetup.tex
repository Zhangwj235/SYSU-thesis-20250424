% !TeX root = ./thesis.tex

\sysusetup{
	title                   = {中山大学学位论文},
	title*                  = {A thesis template for Sun Yat-sen University},
	author                  = {Zhangwj235},
	author*                 = {Jack},
	student-id              = {00000000},
	department              = {xx学院},
	department*             = {xx},
	speciality              = {专业},
	speciality*             = {Major},
	supervisor              = {xx~教授}, %{李腾~教授 , 黄新飞~教授},
	supervisor*             = {Prof. xx},%{Prof. Li Teng, Prof. Huang Xinfei},
	keywords                = {a, b,c},
	keywords*               = {a, b,c},
	% date                    = {2025-05-01},   % 默认为今日
	cover-title             = twoline,   % 默认为 oneline
	cover-title-firstline   = {xxxx},
	cover-title-secondline  = {xxxx},
	cover-title-firstline*  = {xxxx},
	cover-title-secondline* = {xxxx},
	% color                   = black,   % 默认为 sysugreen
	% print                   = twoside,   % 默认为 oneside
	% number                  = chinese,   % 默认为 arabic
}

% 设置英文字体
\setmainfont{Times New Roman}
\setsansfont{Arial}
\setmonofont{inconsolata}
\usepackage{xeCJK}    %声明宏包,主要用于支持在XeTeX和LuaTeX这两种TeX引擎下的CJK(中文、日文、韩文)排版
\usepackage{fontspec} %提供了一个配置和使用不同字体的接口 
\xeCJKsetup{AutoFallBack=true}
\setCJKmainfont{FandolSong}[FallBack=Simsun.ttf]

% 设置数学字体及相关格式
\unimathsetup{
	math-style=ISO,
	% partial=upright,
	% nabla=italic,
}

\setmathfont{XITSMath-Regular}[
	Extension=.otf,
	BoldFont=XITSMath-Bold,
]
\setmathfont{XITSMath-Regular}[
	Extension=.otf,
	range={cal,bfcal},
	StylisticSet=1,
]

% \setmathfont{NewCMMath-Book}[
%     Extension=.otf,
%     BoldFont=NewCMMath-Bold.otf,
% ]
% \setmathfont{NewCMMath-Book}[
%     Extension=.otf,
%     range={scr,bfscr},
%     StylisticSet=1,
% ]

% 加载额外的宏包

% 定理类环境宏包
\usepackage{aliascnt}
\usepackage{amsthm}

% 插图
\usepackage{graphicx}

% 三线表
\usepackage{booktabs}

% 图片并排
\usepackage{subcaption}

% 跨页表格
\usepackage{longtable}
\usepackage{makecell}
\usepackage[graphicx]{realboxes}
\usepackage[figuresright]{rotating}  %表格横置
\def\sym#1{\ifmmode^{#1}\else\(^{#1}\)\fi}  %上标
\usepackage{booktabs} %三线式表格
\usepackage{threeparttable}
\usepackage{float} %固定位置[H]时用
\usepackage{caption} %表注
\newcommand\fnote[1]{\captionsetup{font=small}\caption*{#1}}
\usepackage{subcaption}
\usepackage{tabularx}
\usepackage{multirow}

% 参考文献使用 BibTeX + natbib 宏包
% 顺序编码制
%\usepackage[sort&compress]{gbt7714}
%\bibliographystyle{gbt7714-numerical}
% 著者-出版年制
%\bibliographystyle{gbt7714-author-year}

% 参考文献使用 BibLaTeX 宏包
% \usepackage[backend=biber,style=gb7714-2015]{biblatex}
%\usepackage[backend=biber,style=gb7714-2015ay]{biblatex}
\usepackage[backend=biber,style=gb7714-2015ay,uniquelist=false,maxcitenames=2,mincitenames=1,nohashothers=false]{biblatex}

\DefineBibliographyStrings{english}{
        andincite         = { \& },
        andincitecn       = {和},
}
% 声明 BibLaTeX 的数据库
\addbibresource{bib/sysu.bib}

% hyperref 宏包在最后调用
\usepackage{hyperref}

% 配置图片的默认目录
\graphicspath{{figures/}{pictures/}}
